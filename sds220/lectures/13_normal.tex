\documentclass[10pt]{article}\usepackage[]{graphicx}\usepackage[]{color}
%% maxwidth is the original width if it is less than linewidth
%% otherwise use linewidth (to make sure the graphics do not exceed the margin)
\makeatletter
\def\maxwidth{ %
  \ifdim\Gin@nat@width>\linewidth
    \linewidth
  \else
    \Gin@nat@width
  \fi
}
\makeatother

\definecolor{fgcolor}{rgb}{0.345, 0.345, 0.345}
\newcommand{\hlnum}[1]{\textcolor[rgb]{0.686,0.059,0.569}{#1}}%
\newcommand{\hlstr}[1]{\textcolor[rgb]{0.192,0.494,0.8}{#1}}%
\newcommand{\hlcom}[1]{\textcolor[rgb]{0.678,0.584,0.686}{\textit{#1}}}%
\newcommand{\hlopt}[1]{\textcolor[rgb]{0,0,0}{#1}}%
\newcommand{\hlstd}[1]{\textcolor[rgb]{0.345,0.345,0.345}{#1}}%
\newcommand{\hlkwa}[1]{\textcolor[rgb]{0.161,0.373,0.58}{\textbf{#1}}}%
\newcommand{\hlkwb}[1]{\textcolor[rgb]{0.69,0.353,0.396}{#1}}%
\newcommand{\hlkwc}[1]{\textcolor[rgb]{0.333,0.667,0.333}{#1}}%
\newcommand{\hlkwd}[1]{\textcolor[rgb]{0.737,0.353,0.396}{\textbf{#1}}}%
\let\hlipl\hlkwb

\usepackage{framed}
\makeatletter
\newenvironment{kframe}{%
 \def\at@end@of@kframe{}%
 \ifinner\ifhmode%
  \def\at@end@of@kframe{\end{minipage}}%
  \begin{minipage}{\columnwidth}%
 \fi\fi%
 \def\FrameCommand##1{\hskip\@totalleftmargin \hskip-\fboxsep
 \colorbox{shadecolor}{##1}\hskip-\fboxsep
     % There is no \\@totalrightmargin, so:
     \hskip-\linewidth \hskip-\@totalleftmargin \hskip\columnwidth}%
 \MakeFramed {\advance\hsize-\width
   \@totalleftmargin\z@ \linewidth\hsize
   \@setminipage}}%
 {\par\unskip\endMakeFramed%
 \at@end@of@kframe}
\makeatother

\definecolor{shadecolor}{rgb}{.97, .97, .97}
\definecolor{messagecolor}{rgb}{0, 0, 0}
\definecolor{warningcolor}{rgb}{1, 0, 1}
\definecolor{errorcolor}{rgb}{1, 0, 0}
\newenvironment{knitrout}{}{} % an empty environment to be redefined in TeX

\usepackage{alltt}

\usepackage{amsmath,amssymb,amsthm}
\usepackage{fancyhdr,url,hyperref}
\usepackage{graphicx,xspace}
\usepackage{subfigure}
\usepackage{tikz}
\usetikzlibrary{arrows,decorations.pathmorphing,backgrounds,positioning,fit,through}

\oddsidemargin 0in  %0.5in
\topmargin     0in
\leftmargin    0in
\rightmargin   0in
\textheight    9in
\textwidth     6in %6in
%\headheight    0in
%\headsep       0in
%\footskip      0.5in

\newtheorem{thm}{Theorem}
\newtheorem{cor}[thm]{Corollary}
\newtheorem{obs}{Observation}
\newtheorem{lemma}{Lemma}
\newtheorem{claim}{Claim}
\newtheorem{definition}{Definition}
\newtheorem{question}{Question}
\newtheorem{answer}{Answer}
\newtheorem{problem}{Problem}
\newtheorem{solution}{Solution}
\newtheorem{conjecture}{Conjecture}

\pagestyle{fancy}

\lhead{\textsc{Prof. McNamara}}
\chead{\textsc{SDS/MTH 220: Lecture notes}}
\lfoot{}
\cfoot{}
%\cfoot{\thepage}
\rfoot{}
\renewcommand{\headrulewidth}{0.2pt}
\renewcommand{\footrulewidth}{0.0pt}

\newcommand{\ans}{\vspace{0.25in}}
\newcommand{\R}{{\sf R}\xspace}
\newcommand{\cmd}[1]{\texttt{#1}}

\rhead{\textsc{March 1, 2017}}
\IfFileExists{upquote.sty}{\usepackage{upquote}}{}
\begin{document}

\paragraph{Agenda}
\begin{enumerate}
  \itemsep0em
  \item Central Limit Theorem
  \item Normal Distribution
\end{enumerate}

\paragraph{Warmup: Distribution of Height on OkCupid}

Consider the \href{http://cdn.okcimg.com/blog/lies/MaleHeightDistribution.png}{distribution of reported male height} for users of the online dating site OkCupid. 

\begin{enumerate}
  \item What observations can you make from this data graphic?
  \vspace{0.5in}
\end{enumerate}

\paragraph{Example: MLB Batting Averages}

In 1941, \href{https://www.google.com/search?q=ted+williams}{Ted Williams} of the Boston Red Sox hit .406, famously getting 6 hits in 8 at-bats on the last day of the season. No player in Major League Baseball has hit .400 since. Among the closest attempts was made by \href{http://deadspin.com/this-picture-of-george-brett-inspired-that-lorde-song-1478665015}{George Brett} of the Kansas City Royals in 1980, when Brett hit .390. When viewed in relation to his peers, whose performance was more impressive?

\begin{knitrout}
\definecolor{shadecolor}{rgb}{0.969, 0.969, 0.969}\color{fgcolor}\begin{kframe}
\begin{alltt}
\hlkwd{require}\hlstd{(Lahman)}
\hlkwd{require}\hlstd{(mosaic)}
\hlstd{mlb} \hlkwb{<-} \hlstd{Batting} \hlopt
  \hlkwd{mutate}\hlstd{(}\hlkwc{BAvg} \hlstd{= H} \hlopt{/} \hlstd{AB)} \hlopt
  \hlkwd{filter}\hlstd{(yearID} \hlopt \hlkwd{c}\hlstd{(}\hlnum{1941}\hlstd{,} \hlnum{1980}\hlstd{)} \hlopt{&} \hlstd{AB} \hlopt{>} \hlnum{400}\hlstd{)}
\hlstd{mlb} \hlopt
  \hlkwd{filter}\hlstd{(BAvg} \hlopt{>} \hlnum{.36}\hlstd{)} \hlopt
  \hlkwd{select}\hlstd{(playerID, yearID, BAvg)}
\end{alltt}
\begin{verbatim}
##    playerID yearID      BAvg
## 1 willite01   1941 0.4057018
## 2 brettge01   1980 0.3897550
\end{verbatim}
\end{kframe}
\end{knitrout}




\begin{enumerate}
  \itemsep1in
  \item Use the information below to calculate a $z$-score for both Williams in 1941 and Brett in 1980. 

\begin{knitrout}
\definecolor{shadecolor}{rgb}{0.969, 0.969, 0.969}\color{fgcolor}\begin{kframe}
\begin{alltt}
\hlstd{mlb} \hlopt
  \hlkwd{group_by}\hlstd{(yearID)} \hlopt
  \hlkwd{summarize}\hlstd{(}\hlkwc{N} \hlstd{=} \hlkwd{n}\hlstd{(),} \hlkwc{mean_BAvg} \hlstd{=} \hlkwd{mean}\hlstd{(BAvg),} \hlkwc{sd_BAvg} \hlstd{=} \hlkwd{sd}\hlstd{(BAvg))}
\end{alltt}
\begin{verbatim}
## # A tibble: 2 × 4
##   yearID     N mean_BAvg    sd_BAvg
##    <int> <int>     <dbl>      <dbl>
## 1   1941    98 0.2806367 0.03279026
## 2   1980   148 0.2788247 0.02757441
\end{verbatim}
\end{kframe}
\end{knitrout}

  \item Whose performance do you think was more remarkable in the context of his peers? Why? What assumptions are you making? 
\vspace{0.75in}

\end{enumerate}





\paragraph{The Empirical Rule for Normal Distributions}

\begin{figure}
  \includegraphics[width=\textwidth]{../gfx/Standard_deviation_diagram}
\end{figure}

For any normally distributed variable:
\begin{itemize}
  \itemsep0in
	\item About 68\% of the distribution is contained within 1 standard deviation of the mean.
	\item About 95\% of the distribution is contained within 2 standard deviations of the mean.
	\item About 99.7\% of the distribution is contained within 3 standard deviations of the mean.
	\item About 38\% of the distribution is contained within 0.5 standard deviations of the mean.
	\begin{itemize}
		\item In other words, the middle 38\% of the distribution is about 1 standard deviation wide.
	\end{itemize}
\end{itemize}

\paragraph{Sample Calculations}
\begin{enumerate}
	\item What percentage of the distribution is less than 2 standard deviations above the mean?
	\begin{itemize}
		\item By the rule, about 95\% of the population is within two standard deviations of the mean. By symmetry, half
of those are above the mean, and half below it. Thus, we estimate that about $95/2 = 47.5\%$ is less than 2
standard deviations above the mean.
		\item From the picture, we can calculate the area as about $34.1\% + 13.6\% = 47.7\%$
	\end{itemize}
	\item Assume that the distribution of heights of adult women is approximately normal with mean 64 inches and standard
deviation 2.5 inches.
	\begin{enumerate}
    \itemsep0.5in
		\item What percentage of women are taller than 5'9"?
		

		\item Between what heights do the middle 95\% of women fall?

		
		
		\item What percentage of women are shorter than 61.5 inches?

    \item A professor claims that about 51\% of women are between 61.5 and 65.25 inches tall. Is this claim accurate?
	\end{enumerate}
\end{enumerate}



% 
% \newpage
% 
% \subsection*{Instructor's Notes}
% 
% \paragraph{Okcupid}
% 
% Start class by projecting the distribution of self-reported heights on OKcupid and ask the class to describe what they see.  There are two interesting features: the whole shift of the distribution from the US distribution (is that real or not?) and the fact that men tend to also nudge themselves up to 6 feet.
% 
% \paragraph{Lecture}
% 
% \begin{itemize}
%   \item Central Limit Theorem:
%   \begin{itemize}
%     \item The distribution of the sample mean (i.e. the sampling distribution of the mean) will be approximately normal for reasonably large $n$ (at least 30)
%     \item Provides a mathematical approximation to the simulated null distributions with which we have been working. Consider the practical difficulties of simulating null distributions with a computer!
%   \end{itemize}
%   \item Normal distribution has two parameters!
%   $$
%     N(\mu, \sigma), \qquad \text{ has density function } f(x; \mu, \sigma) = \frac{1}{\sqrt{2 \pi} \sigma} \exp \left( \frac{(x - \mu)^2}{2 \sigma^2} \right)
%   $$
%   \item $N(0,1)$: Standard Normal distribution
%   \item Consider $N(\mu, \sigma)$. What is the distribution of $N(\mu, \sigma) - \mu$? What is the distribution of $N(0, \sigma)/ \sigma$?
%   \item If $x \sim N(\mu, \sigma)$, then $z = \frac{x - \mu}{\sigma} \sim N(0,1)$.
% 
% \end{itemize}
% 
% \paragraph{Z-score example}
% 
% SAT scores are distributed nearly normally with mean 1500 and sd 300.  ACT scores are distributed nearly normally with mean 21 and sd 5.  A college admissions officer wants to determine which of the two applicants scored better on their standardized test with respect to the other test takers: Pam, who eared an 1800 on her SAT, or Jim, who scored a 24 on his ACT? \\
% 
% Pam's score is $\frac{1800-1500}{300} = 1$ sd above the mean.
% 
% Jim's score is $\frac{24-21}{5} = 0.6$ sd above the mean.




\end{document}
