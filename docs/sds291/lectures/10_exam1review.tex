\documentclass[10pt]{article}

\usepackage{amsmath,amssymb,amsthm}
\usepackage{fancyhdr,url,hyperref}
\usepackage{graphicx,xspace}
\usepackage{tikz}
\usetikzlibrary{shapes,arrows,decorations.pathmorphing,backgrounds,positioning,fit,through}

\oddsidemargin 0in  %0.5in
\topmargin     0in
\leftmargin    0in
\rightmargin   0in
\textheight    9in
\textwidth     6in %6in
%\headheight    0in
%\headsep       0in
%\footskip      0.5in

\newtheorem{thm}{Theorem}
\newtheorem{cor}[thm]{Corollary}
\newtheorem{obs}{Observation}
\newtheorem{lemma}{Lemma}
\newtheorem{claim}{Claim}
\newtheorem{definition}{Definition}
\newtheorem{question}{Question}
\newtheorem{answer}{Answer}
\newtheorem{problem}{Problem}
\newtheorem{solution}{Solution}
\newtheorem{conjecture}{Conjecture}

\pagestyle{fancy}

\lhead{\textsc{Prof. McNamara}}
\chead{\textsc{SDS/MTH 291: Lecture notes}}
\lfoot{}
\cfoot{}
%\cfoot{\thepage}
\rfoot{}
\renewcommand{\headrulewidth}{0.2pt}
\renewcommand{\footrulewidth}{0.0pt}

\newcommand{\ans}{\vspace{0.25in}}
\newcommand{\R}{{\sf R}\xspace}
\newcommand{\cmd}[1]{\texttt{#1}}
\DeclareMathOperator{\Ex}{\mathbb{E}}
\DeclareMathOperator{\Var}{\text{Var}}

\rhead{\textsc{October 20, 2016}}

\usepackage{Sweave}
\begin{document}
\Sconcordance{concordance:10_exam1review.tex:10_exam1review.Rnw:%
1 49 1 1 0 40 1}


\paragraph{Agenda}
\begin{enumerate}
  \itemsep0em
  \item Putting it all together
  \item Regression summary lab?
\end{enumerate}

\paragraph{Pulling it all together}
We use models to help us simplify the world. Linear models assume a linear structure in 2 or many dimensions, and can be used to summarize existing data or predict values for new data. 

Using data, we make estimates of model parameters (statistics) and try to determine how good our estimates are. We're always assuming our data is a random sample of the larger population of interest. 
\\
\\ Relationship between the population, the sample, the parameters and the statistics:
\vspace{1.5in}
\\
Lets try to categorize some of the tasks we have learned how to perform. Where in the CFAU process do these fit?
\\ Checking conditions, performing hypothesis tests on coefficients, interpreting coefficient values, finding confidence intervals for means, finding confidence intervals for specific values (prediction intervals), performing transformations, checking for outliers, looking at $R^2$, looking at $R^2_{adj}$. 
\\
What are some more tasks we have performed?
\\
\\
Choose
\vspace{1in}
\\
Fit
\vspace{1in}
\\
Assess
\vspace{1in}
\\
Use
\vspace{1in}





\end{document}
