\documentclass[10pt]{article}\usepackage[]{graphicx}\usepackage[]{color}
%% maxwidth is the original width if it is less than linewidth
%% otherwise use linewidth (to make sure the graphics do not exceed the margin)
\makeatletter
\def\maxwidth{ %
  \ifdim\Gin@nat@width>\linewidth
    \linewidth
  \else
    \Gin@nat@width
  \fi
}
\makeatother

\definecolor{fgcolor}{rgb}{0.345, 0.345, 0.345}
\newcommand{\hlnum}[1]{\textcolor[rgb]{0.686,0.059,0.569}{#1}}%
\newcommand{\hlstr}[1]{\textcolor[rgb]{0.192,0.494,0.8}{#1}}%
\newcommand{\hlcom}[1]{\textcolor[rgb]{0.678,0.584,0.686}{\textit{#1}}}%
\newcommand{\hlopt}[1]{\textcolor[rgb]{0,0,0}{#1}}%
\newcommand{\hlstd}[1]{\textcolor[rgb]{0.345,0.345,0.345}{#1}}%
\newcommand{\hlkwa}[1]{\textcolor[rgb]{0.161,0.373,0.58}{\textbf{#1}}}%
\newcommand{\hlkwb}[1]{\textcolor[rgb]{0.69,0.353,0.396}{#1}}%
\newcommand{\hlkwc}[1]{\textcolor[rgb]{0.333,0.667,0.333}{#1}}%
\newcommand{\hlkwd}[1]{\textcolor[rgb]{0.737,0.353,0.396}{\textbf{#1}}}%
\let\hlipl\hlkwb

\usepackage{framed}
\makeatletter
\newenvironment{kframe}{%
 \def\at@end@of@kframe{}%
 \ifinner\ifhmode%
  \def\at@end@of@kframe{\end{minipage}}%
  \begin{minipage}{\columnwidth}%
 \fi\fi%
 \def\FrameCommand##1{\hskip\@totalleftmargin \hskip-\fboxsep
 \colorbox{shadecolor}{##1}\hskip-\fboxsep
     % There is no \\@totalrightmargin, so:
     \hskip-\linewidth \hskip-\@totalleftmargin \hskip\columnwidth}%
 \MakeFramed {\advance\hsize-\width
   \@totalleftmargin\z@ \linewidth\hsize
   \@setminipage}}%
 {\par\unskip\endMakeFramed%
 \at@end@of@kframe}
\makeatother

\definecolor{shadecolor}{rgb}{.97, .97, .97}
\definecolor{messagecolor}{rgb}{0, 0, 0}
\definecolor{warningcolor}{rgb}{1, 0, 1}
\definecolor{errorcolor}{rgb}{1, 0, 0}
\newenvironment{knitrout}{}{} % an empty environment to be redefined in TeX

\usepackage{alltt}

\usepackage{amsmath,amssymb,amsthm}
\usepackage{fancyhdr,url,hyperref}
\usepackage{graphicx,xspace}
\usepackage{tikz}
\usetikzlibrary{shapes,arrows,decorations.pathmorphing,backgrounds,positioning,fit,through}
\oddsidemargin 0in  %0.5in
\topmargin     0in
\leftmargin    0in
\rightmargin   0in
\textheight    9in
\textwidth     6in %6in
%\headheight    0in
%\headsep       0in
%\footskip      0.5in

\newtheorem{thm}{Theorem}
\newtheorem{cor}[thm]{Corollary}
\newtheorem{obs}{Observation}
\newtheorem{lemma}{Lemma}
\newtheorem{claim}{Claim}
\newtheorem{definition}{Definition}
\newtheorem{question}{Question}
\newtheorem{answer}{Answer}
\newtheorem{problem}{Problem}
\newtheorem{solution}{Solution}
\newtheorem{conjecture}{Conjecture}

\pagestyle{fancy}

\lhead{\textsc{Prof. McNamara}}
\chead{\textsc{SDS/MTH 291: Lecture notes}}
\lfoot{}
\cfoot{}
%\cfoot{\thepage}
\rfoot{}
\renewcommand{\headrulewidth}{0.2pt}
\renewcommand{\footrulewidth}{0.0pt}

\newcommand{\ans}{\vspace{0.25in}}
\newcommand{\R}{{\sf R}\xspace}
\newcommand{\cmd}[1]{\texttt{#1}}
\DeclareMathOperator{\Ex}{\mathbb{E}}
\DeclareMathOperator{\Var}{\text{Var}}
\DeclareMathOperator{\X}{\mathbf{X}}
\DeclareMathOperator{\Hatmat}{\mathbf{H}}

\rhead{\textsc{November 17, 2016}}
\IfFileExists{upquote.sty}{\usepackage{upquote}}{}
\begin{document}

\paragraph{Agenda}
\begin{enumerate}
  \itemsep0em
  \item More ANOVA
  \item ANOVA lab
\end{enumerate}



\paragraph{ANOVA}
Statisticians overload the term ANOVA. We use it when we are doing nested F-tests, when we are testing the significance of a categorical variable within a larger linear regression model, and we use it when we are just predicting a quantitative variable with a categorical one.

% Individual t-tests in regression output give us a way to test the statistical significant of individual terms in our model. But what if we want to test the significance of the contribution to the model by a \emph{subset} of the predictors.
%   \begin{itemize}
% 		\item $H_0$: $\beta_i=0$ for all predictors in the subset
% 		\item $H_{A}$: at least one $\beta_i \neq 0$
% 		$$
% 			F = \frac{(SSR_{full} - SSR_{reduced}) / (\text{\# of predictors tested})}{SSE_{full} / (n-k-1)} ,
% 		$$
% 		where $k$ is the \# of predictors in the full model
% 		\item Use \texttt{anova} command in R, being careful that terms in the model are \emph{nested}.
% 	\end{itemize}

Just to get some intuition, lets compare the results from a few different approaches to using ANOVA. 
	
\begin{knitrout}
\definecolor{shadecolor}{rgb}{0.969, 0.969, 0.969}\color{fgcolor}\begin{kframe}
\begin{alltt}
\hlkwd{require}\hlstd{(mosaic)}
\hlstd{bloodp} \hlkwb{=} \hlkwd{read.csv}\hlstd{(}\hlstr{"http://www.math.smith.edu/~bbaumer/mth247/labs/bloodpress.csv"}\hlstd{)}
\hlstd{m1} \hlkwb{=} \hlkwd{lm}\hlstd{(BP} \hlopt{~} \hlstd{Weight,} \hlkwc{data}\hlstd{=bloodp)}
\hlkwd{anova}\hlstd{(m1)}
\end{alltt}
\begin{verbatim}
## Analysis of Variance Table
## 
## Response: BP
##           Df Sum Sq Mean Sq F value    Pr(>F)    
## Weight     1 505.47  505.47  166.86 1.528e-10 ***
## Residuals 18  54.53    3.03                      
## ---
## Signif. codes:  0 '***' 0.001 '**' 0.01 '*' 0.05 '.' 0.1 ' ' 1
\end{verbatim}
\begin{alltt}
\hlstd{m2} \hlkwb{=} \hlkwd{lm}\hlstd{(BP} \hlopt{~} \hlstd{Age,} \hlkwc{data}\hlstd{=bloodp)}
\hlkwd{anova}\hlstd{(m2)}
\end{alltt}
\begin{verbatim}
## Analysis of Variance Table
## 
## Response: BP
##           Df Sum Sq Mean Sq F value   Pr(>F)   
## Age        1 243.27 243.266  13.825 0.001574 **
## Residuals 18 316.73  17.596                    
## ---
## Signif. codes:  0 '***' 0.001 '**' 0.01 '*' 0.05 '.' 0.1 ' ' 1
\end{verbatim}
\begin{alltt}
\hlstd{m3} \hlkwb{=} \hlkwd{aov}\hlstd{(BP}\hlopt{~} \hlstd{Weight} \hlopt{+} \hlstd{Age,} \hlkwc{data}\hlstd{=bloodp)}
\hlkwd{summary}\hlstd{(m3)}
\end{alltt}
\begin{verbatim}
##             Df Sum Sq Mean Sq F value   Pr(>F)    
## Weight       1  505.5   505.5  1781.3  < 2e-16 ***
## Age          1   49.7    49.7   175.2 2.22e-10 ***
## Residuals   17    4.8     0.3                     
## ---
## Signif. codes:  0 '***' 0.001 '**' 0.01 '*' 0.05 '.' 0.1 ' ' 1
\end{verbatim}
\begin{alltt}
\hlstd{m4} \hlkwb{=} \hlkwd{aov}\hlstd{(BP}\hlopt{~} \hlstd{Age} \hlopt{+}  \hlstd{Weight,} \hlkwc{data}\hlstd{=bloodp)}
\hlkwd{summary}\hlstd{(m4)}
\end{alltt}
\begin{verbatim}
##             Df Sum Sq Mean Sq F value   Pr(>F)    
## Age          1 243.27  243.27   857.3 5.48e-16 ***
## Weight       1 311.91  311.91  1099.2  < 2e-16 ***
## Residuals   17   4.82    0.28                     
## ---
## Signif. codes:  0 '***' 0.001 '**' 0.01 '*' 0.05 '.' 0.1 ' ' 1
\end{verbatim}
\begin{alltt}
\hlstd{m5} \hlkwb{=} \hlkwd{aov}\hlstd{(BP}\hlopt{~}\hlstd{.,} \hlkwc{data}\hlstd{=bloodp)}
\hlkwd{summary}\hlstd{(m5)}
\end{alltt}
\begin{verbatim}
##             Df Sum Sq Mean Sq  F value   Pr(>F)    
## Age          1 243.27  243.27 1466.914 9.38e-15 ***
## Weight       1 311.91  311.91 1880.844 1.89e-15 ***
## BSA          1   1.77    1.77   10.660  0.00615 ** 
## Dur          1   0.34    0.34    2.021  0.17871    
## Pulse        1   0.12    0.12    0.742  0.40458    
## Stress       1   0.44    0.44    2.666  0.12649    
## Residuals   13   2.16    0.17                      
## ---
## Signif. codes:  0 '***' 0.001 '**' 0.01 '*' 0.05 '.' 0.1 ' ' 1
\end{verbatim}
\begin{alltt}
\hlstd{m6} \hlkwb{=} \hlkwd{aov}\hlstd{(BP}\hlopt{~}\hlstd{Stress}\hlopt{+}\hlstd{Pulse}\hlopt{+}\hlstd{Dur}\hlopt{+}\hlstd{BSA}\hlopt{+}\hlstd{Weight}\hlopt{+}\hlstd{Age,} \hlkwc{data}\hlstd{=bloodp)}
\hlkwd{summary}\hlstd{(m6)}
\end{alltt}
\begin{verbatim}
##             Df Sum Sq Mean Sq  F value   Pr(>F)    
## Stress       1  15.04   15.04   90.714 3.16e-07 ***
## Pulse        1 306.93  306.93 1850.831 2.09e-15 ***
## Dur          1   0.72    0.72    4.337   0.0576 .  
## BSA          1 173.87  173.87 1048.444 8.15e-14 ***
## Weight       1  27.95   27.95  168.533 8.09e-09 ***
## Age          1  33.33   33.33  200.986 2.76e-09 ***
## Residuals   13   2.16    0.17                      
## ---
## Signif. codes:  0 '***' 0.001 '**' 0.01 '*' 0.05 '.' 0.1 ' ' 1
\end{verbatim}
\end{kframe}
\end{knitrout}

\paragraph{ANOVA lab}
For your convenience, here is the code from the ANOVA lab
\begin{knitrout}
\definecolor{shadecolor}{rgb}{0.969, 0.969, 0.969}\color{fgcolor}\begin{kframe}
\begin{alltt}
\hlkwd{require}\hlstd{(mosaic)}
\hlkwd{require}\hlstd{(Stat2Data)}

\hlkwd{data}\hlstd{(FruitFlies)}
\hlkwd{head}\hlstd{(FruitFlies)}

\hlkwd{tally}\hlstd{(}\hlopt{~}\hlstd{Treatment,} \hlkwc{data}\hlstd{=FruitFlies)}

\hlcom{# Set the reference level}
\hlstd{FruitFlies} \hlkwb{=} \hlstd{FruitFlies} \hlopt
  \hlkwd{mutate}\hlstd{(}\hlkwc{Treatment} \hlstd{=} \hlkwd{relevel}\hlstd{(Treatment,} \hlkwc{ref}\hlstd{=}\hlstr{"none"}\hlstd{))}

\hlstd{d1plot} \hlkwb{=} \hlkwd{dotPlot}\hlstd{(}\hlopt{~}\hlstd{Longevity,} \hlkwc{data}\hlstd{=FruitFlies)}
\hlstd{d2plot} \hlkwb{=} \hlkwd{dotPlot}\hlstd{(}\hlopt{~}\hlstd{Longevity} \hlopt{|} \hlstd{Treatment,} \hlkwc{data}\hlstd{=FruitFlies,} \hlkwc{layout}\hlstd{=}\hlkwd{c}\hlstd{(}\hlnum{1}\hlstd{,}\hlnum{5}\hlstd{))}
\hlcom{# arrange the two plots horizontally}
\hlkwd{print}\hlstd{(d1plot,} \hlkwc{position}\hlstd{=}\hlkwd{c}\hlstd{(}\hlnum{0}\hlstd{,} \hlnum{0}\hlstd{,} \hlnum{0.5}\hlstd{,} \hlnum{1}\hlstd{),} \hlkwc{more}\hlstd{=}\hlnum{TRUE}\hlstd{)}
\hlkwd{print}\hlstd{(d2plot,} \hlkwc{position}\hlstd{=}\hlkwd{c}\hlstd{(}\hlnum{0.5}\hlstd{,} \hlnum{0}\hlstd{,} \hlnum{1}\hlstd{,} \hlnum{1}\hlstd{))}

\hlstd{b1plot} \hlkwb{=} \hlkwd{bwplot}\hlstd{(}\hlopt{~}\hlstd{Longevity,} \hlkwc{data}\hlstd{=FruitFlies)}
\hlstd{b2plot} \hlkwb{=} \hlkwd{bwplot}\hlstd{(}\hlopt{~}\hlstd{Longevity} \hlopt{|} \hlstd{Treatment,} \hlkwc{data}\hlstd{=FruitFlies,} \hlkwc{layout}\hlstd{=}\hlkwd{c}\hlstd{(}\hlnum{1}\hlstd{,}\hlnum{5}\hlstd{))}
\hlcom{# arrange the two plots horizontally}
\hlkwd{print}\hlstd{(b1plot,} \hlkwc{position}\hlstd{=}\hlkwd{c}\hlstd{(}\hlnum{0}\hlstd{,} \hlnum{0}\hlstd{,} \hlnum{0.5}\hlstd{,} \hlnum{1}\hlstd{),} \hlkwc{more}\hlstd{=}\hlnum{TRUE}\hlstd{)}
\hlkwd{print}\hlstd{(b2plot,} \hlkwc{position}\hlstd{=}\hlkwd{c}\hlstd{(}\hlnum{0.5}\hlstd{,} \hlnum{0}\hlstd{,} \hlnum{1}\hlstd{,} \hlnum{1}\hlstd{))}

\hlstd{gstats} \hlkwb{=} \hlkwd{favstats}\hlstd{(Longevity} \hlopt{~} \hlstd{Treatment,} \hlkwc{data}\hlstd{=FruitFlies)}
\hlstd{gstats}

\hlstd{fm.null} \hlkwb{=} \hlkwd{lm}\hlstd{(Longevity} \hlopt{~} \hlnum{1}\hlstd{,} \hlkwc{data}\hlstd{=FruitFlies)}
\hlkwd{summary}\hlstd{(fm.null)}
\hlkwd{mean}\hlstd{(}\hlopt{~}\hlstd{Longevity,} \hlkwc{data}\hlstd{=FruitFlies)}
\hlstd{fitted} \hlkwb{=} \hlkwd{fitted.values}\hlstd{(fm.null)}
\hlkwd{ladd}\hlstd{(}\hlkwd{panel.abline}\hlstd{(}\hlkwc{v}\hlstd{=}\hlkwd{coef}\hlstd{(fm.null),} \hlkwc{col}\hlstd{=}\hlstr{"red"}\hlstd{,} \hlkwc{lwd}\hlstd{=}\hlnum{3}\hlstd{),} \hlkwc{plot}\hlstd{=d2plot)}

\hlstd{fm.aov} \hlkwb{=} \hlkwd{aov}\hlstd{(Longevity} \hlopt{~} \hlstd{Treatment,} \hlkwc{data}\hlstd{=FruitFlies)}
\hlkwd{summary}\hlstd{(fm.aov)}
\hlkwd{model.tables}\hlstd{(fm.aov)}

\hlstd{fm.ref} \hlkwb{=} \hlkwd{lm}\hlstd{(Longevity} \hlopt{~} \hlstd{Treatment,} \hlkwc{data}\hlstd{=FruitFlies)}
\hlkwd{summary}\hlstd{(fm.ref)}
\hlkwd{anova}\hlstd{(fm.ref)}
\hlkwd{mean}\hlstd{(Longevity}\hlopt{~}\hlstd{Treatment,} \hlkwc{data} \hlstd{= FruitFlies)} \hlopt{-} \hlkwd{mean}\hlstd{(}\hlopt{~}\hlstd{Longevity,} \hlkwc{data} \hlstd{= FruitFlies)}

\hlkwd{dotPlot}\hlstd{(}\hlopt{~}\hlstd{Longevity} \hlopt{|} \hlstd{Treatment,} \hlkwc{data}\hlstd{=FruitFlies,} \hlkwc{layout}\hlstd{=}\hlkwd{c}\hlstd{(}\hlnum{1}\hlstd{,}\hlnum{5}\hlstd{),} \hlkwc{panel}\hlstd{=}\hlkwa{function}\hlstd{(}\hlkwc{x}\hlstd{,}\hlkwc{...}\hlstd{)\{}
  \hlkwd{panel.dotPlot}\hlstd{(x, ...)}
  \hlstd{fitted} \hlkwb{=} \hlkwd{mean}\hlstd{(x)}
  \hlkwd{panel.abline}\hlstd{(}\hlkwc{v}\hlstd{=fitted,} \hlkwc{col}\hlstd{=}\hlstr{"red"}\hlstd{)}
\hlstd{\})}

\hlkwd{par}\hlstd{(}\hlkwc{mfrow}\hlstd{=}\hlkwd{c}\hlstd{(}\hlnum{2}\hlstd{,}\hlnum{2}\hlstd{))}
\hlkwd{plot}\hlstd{(fm.aov)}
\hlkwd{par}\hlstd{(}\hlkwc{mfrow}\hlstd{=}\hlkwd{c}\hlstd{(}\hlnum{1}\hlstd{,}\hlnum{1}\hlstd{))}

\hlkwd{sd}\hlstd{(Longevity}\hlopt{~}\hlstd{Treatment,} \hlkwc{data}\hlstd{=FruitFlies)}
\hlkwd{max}\hlstd{(}\hlkwd{sd}\hlstd{(Longevity}\hlopt{~}\hlstd{Treatment,} \hlkwc{data}\hlstd{=FruitFlies))} \hlopt{/} \hlkwd{min}\hlstd{(}\hlkwd{sd}\hlstd{(Longevity}\hlopt{~}\hlstd{Treatment,} \hlkwc{data}\hlstd{=FruitFlies))}

\hlkwd{data}\hlstd{(Hawks)}
\end{alltt}
\end{kframe}
\end{knitrout}

%Note to self: next time I should include more of the computation/comparison of anova() on a pair of linear models and aov() on a multi-term model. 
\end{document}
